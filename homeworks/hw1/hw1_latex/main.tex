\documentclass{article}
\usepackage{amsmath,amssymb}
\usepackage{graphicx}
\usepackage{enumerate}
\usepackage{hyperref}
\usepackage{subcaption}
\usepackage{caption}
\usepackage{xcolor}
\usepackage{float}

\pagestyle{empty} \addtolength{\textwidth}{1.0in}
\addtolength{\textheight}{0.5in}
\addtolength{\oddsidemargin}{-0.5in}
\addtolength{\evensidemargin}{-0.5in}
\newcommand{\ruleskip}{\bigskip\hrule\bigskip}
\newcommand{\nodify}[1]{{\sc #1}}
\newcommand{\points}[1]{{\textbf{[#1 points]}}}
\newcommand{\subquestionpoints}[1]{{[#1 points]}}
\newenvironment{answer}{{\bf Answer:} \sf }{}%

\newcommand{\bitem}{\begin{list}{$\bullet$}%
{\setlength{\itemsep}{0pt}\setlength{\topsep}{0pt}%
\setlength{\rightmargin}{0pt}}}
\newcommand{\eitem}{\end{list}}

\setlength{\parindent}{0pt} \setlength{\parskip}{0.5ex}
\setlength{\unitlength}{1cm}

\newcommand{\pa}[1]{[[PA: #1]]}

\renewcommand{\Re}{{\mathbb R}}
\newcommand{\E}{{\rm E}}
\begin{document}

\pagestyle{myheadings} \markboth{}{CS 294-158 Deep Unsupervised Learning, Homework 1, Spring 2024}

{\huge
\noindent Homework 1: Autoregressive Models}

\hfill \break
Name: Nolan Wagener
\ruleskip

{\bf Deliverable}: This PDF write-up by {\bf Tuesday February 7th, 23:59pm}.  Your PDF should be generated by simply replacing the placeholder images of this LaTeX document with the appropriate solution images that will be generated automatically when solving each question. The solution images are automatically generated and saved using the accompanying IPython notebook. Your PDF is to be submitted into Gradescope. This PDF already contains a few solution images.  These images will allow you to check your own solution to ensure correctness. Submit this PDF, your iPython notebook, and any other code you wrote to Gradescope!


\vspace{.2in}

%--------------------------------------------------------------------------------
%--------------------------------------------------------------------------------
%--------------------------------------------------------------------------------
\noindent {\bf Question 1: 1D Data}
%--------------------------------------------------------------------------------
%--------------------------------------------------------------------------------
%--------------------------------------------------------------------------------

\begin{enumerate}[(a)]

\item {\bf [10pt] Fitting a Histogram} \\\\
20 parameters for dataset 1 \\
Final test loss for dataset 1: 2.548 nats/dim
\begin{figure}[H]
    \centering
    \begin{subfigure}{0.4\textwidth}
        \centering
        \includegraphics[width=\textwidth]{figures/q1_a_dset1_train_plot.png}
        \caption{Dataset 1: Training curve}
    \end{subfigure}
    \hspace{0.2in}
    \begin{subfigure}{0.4\textwidth}
        \centering
        \includegraphics[width=\textwidth]{figures/q1_a_dset1_learned_dist.png}
        \caption{Dataset 1: Learned distribution}
    \end{subfigure}
\end{figure}
100 parameters for dataset 2 \\
Final test loss for dataset 2: 3.686 nats/dim
\begin{figure}[H]
    \centering
    \begin{subfigure}{0.4\textwidth}
        \centering
        \includegraphics[width=\textwidth]{figures/q1_a_dset2_train_plot.png}
        \caption{Dataset 2: Training curve}
    \end{subfigure}
    \hspace{0.2in}
    \begin{subfigure}{0.4\textwidth}
        \centering
        \includegraphics[width=\textwidth]{figures/q1_a_dset2_learned_dist.png}
        \caption{Dataset 2: Learned distribution}
    \end{subfigure}
\end{figure}

\newpage

\item {\bf [10pt] Fitting Discretized Mixture of Logistics} \\\\
12 parameters for dataset 1 \\
Final test loss for dataset 1: 2.554 nats/dim
\begin{figure}[H]
    \centering
    \begin{subfigure}{0.45\textwidth}
        \centering
        \includegraphics[width=\textwidth]{figures/q1_b_dset1_train_plot.png}
        \caption{Dataset 1: Training curve}
    \end{subfigure}
    \hspace{0.2in}
    \begin{subfigure}{0.45\textwidth}
        \centering
        \includegraphics[width=\textwidth]{figures/q1_b_dset1_learned_dist.png}
        \caption{Dataset 1: Learned distribution}
    \end{subfigure}
\end{figure}
12 parameters for dataset 2 \\
Final test loss for dataset 2: 3.717 nats/dim
\begin{figure}[H]
    \centering
    \begin{subfigure}{0.45\textwidth}
        \centering
        \includegraphics[width=\textwidth]{figures/q1_b_dset2_train_plot.png}
        \caption{Dataset 2: Training curve}
    \end{subfigure}
    \hspace{0.2in}
    \begin{subfigure}{0.45\textwidth}
        \centering
        \includegraphics[width=\textwidth]{figures/q1_b_dset2_learned_dist.png}
        \caption{Dataset 2: Learned distribution}
    \end{subfigure}
\end{figure}
\end{enumerate}


%--------------------------------------------------------------------------------
%--------------------------------------------------------------------------------
%--------------------------------------------------------------------------------
\newpage
\noindent {\bf Question 2: PixelCNNs}
%--------------------------------------------------------------------------------
%--------------------------------------------------------------------------------
%--------------------------------------------------------------------------------

\begin{enumerate}[(a)]
\item {\bf [15pt] PixelCNNs on Shapes and MNIST} \\\\
1M parameters for dataset 1 \\
Final test loss for dataset 1: 0.0429 nats/dim
\begin{figure}[H]
    \centering
    \begin{subfigure}{0.45\textwidth}
        \centering
        \includegraphics[width=\textwidth]{figures/q2_a_dset1_train_plot.png}
        \caption{Dataset 1: Training curve}
    \end{subfigure}
    \hspace{0.2in}
    \begin{subfigure}{0.45\textwidth}
        \centering
        \includegraphics[width=\textwidth]{figures/q2_a_dset1_samples.png}
        \caption{Dataset 1: Samples}
    \end{subfigure}
\end{figure}
1M parameters for dataset 2 \\
Final test loss for dataset 2: 0.0805 nats/dim
\begin{figure}[H]
    \centering
    \begin{subfigure}{0.45\textwidth}
        \centering
        \includegraphics[width=\textwidth]{figures/q2_a_dset2_train_plot.png}
        \caption{Dataset 2: Training curve}
    \end{subfigure}
    \hspace{0.2in}
    \begin{subfigure}{0.45\textwidth}
        \centering
        \includegraphics[width=\textwidth]{figures/q2_a_dset2_samples.png}
        \caption{Dataset 2: Samples}
    \end{subfigure}
\end{figure}

\newpage

\item {\bf [15pt] PixelCNN on Colored Shapes and MNIST: Independent Color Channels} \\\\
10M parameters for dataset 1 \\
Final test loss for dataset 1: 0.0459 nats/dim
\begin{figure}[H]
    \centering
    \begin{subfigure}{0.45\textwidth}
        \centering
        \includegraphics[width=\textwidth]{figures/q2_b_dset1_train_plot.png}
        \caption{Dataset 1: Training curve}
    \end{subfigure}
    \hspace{0.2in}
    \begin{subfigure}{0.45\textwidth}
        \centering
        \includegraphics[width=\textwidth]{figures/q2_b_dset1_samples.png}
        \caption{Dataset 1: Samples}
    \end{subfigure}
\end{figure}
10M parameters for dataset 2 \\
Final test loss for dataset 2: 0.0863 nats/dim
\begin{figure}[H]
    \centering
    \begin{subfigure}{0.45\textwidth}
        \centering
        \includegraphics[width=\textwidth]{figures/q2_b_dset2_train_plot.png}
        \caption{Dataset 2: Training curve}
    \end{subfigure}
    \hspace{0.2in}
    \begin{subfigure}{0.45\textwidth}
        \centering
        \includegraphics[width=\textwidth]{figures/q2_b_dset2_samples.png}
        \caption{Dataset 2: Samples}
    \end{subfigure}
\end{figure}
\end{enumerate}

%--------------------------------------------------------------------------------
%--------------------------------------------------------------------------------
%--------------------------------------------------------------------------------
\newpage
\noindent {\bf Question 3: Causal Transformer -- iGPT}
%--------------------------------------------------------------------------------
%--------------------------------------------------------------------------------
%--------------------------------------------------------------------------------

\begin{enumerate}[(a)]
\item {\bf [15pt] Autoregressive Transformer on Shapes and MNIST} \\\\
43M parameters for dataset 1 \\
Final test loss for dataset 1: 0.0355 nats/dim
\begin{figure}[H]
    \centering
    \begin{subfigure}{0.45\textwidth}
        \centering
        \includegraphics[width=\textwidth]{figures/q3_a_dset1_train_plot.png}
        \caption{Dataset 1: Training curve}
    \end{subfigure}
    \hspace{0.2in}
    \begin{subfigure}{0.45\textwidth}
        \centering
        \includegraphics[width=\textwidth]{figures/q3_a_dset1_samples.png}
        \caption{Dataset 1: Samples}
    \end{subfigure}
\end{figure}
44M parameters for dataset 2 \\
Final test loss for dataset 2: 0.0809 nats/dim
\begin{figure}[H]
    \centering
    \begin{subfigure}{0.45\textwidth}
        \centering
        \includegraphics[width=\textwidth]{figures/q3_a_dset2_train_plot.png}
        \caption{Dataset 2: Training curve}
    \end{subfigure}
    \hspace{0.2in}
    \begin{subfigure}{0.45\textwidth}
        \centering
        \includegraphics[width=\textwidth]{figures/q3_a_dset2_samples.png}
        \caption{Dataset 2: Samples}
    \end{subfigure}
\end{figure}

\newpage

\item {\bf [15pt] Autoregressive Transformer on Colored Shapes and MNIST} \\\\
43M parameters for dataset 1 \\
Final test loss for dataset 1: 0.0538 nats/dim
\begin{figure}[H]
    \centering
    \begin{subfigure}{0.45\textwidth}
        \centering
        \includegraphics[width=\textwidth]{figures/q3_b_dset1_train_plot.png}
        \caption{Dataset 1: Training curve}
    \end{subfigure}
    \hspace{0.2in}
    \begin{subfigure}{0.45\textwidth}
        \centering
        \includegraphics[width=\textwidth]{figures/q3_b_dset1_samples.png}
        \caption{Dataset 1: Samples}
    \end{subfigure}
\end{figure}
43M parameters for dataset 2 \\
Final test loss for dataset 2: 0.0915 nats/dim
\begin{figure}[H]
    \centering
    \begin{subfigure}{0.45\textwidth}
        \centering
        \includegraphics[width=\textwidth]{figures/q3_b_dset2_train_plot.png}
        \caption{Dataset 2: Training curve}
    \end{subfigure}
    \hspace{0.2in}
    \begin{subfigure}{0.45\textwidth}
        \centering
        \includegraphics[width=\textwidth]{figures/q3_b_dset2_samples.png}
        \caption{Dataset 2: Samples}
    \end{subfigure}
\end{figure}

\newpage

\item {\bf [15pt] K, V Caching for Improved Inference} \\\\
\begin{figure}[H]
    \centering
    \includegraphics[width=0.5\textwidth]{figures/q3_c_dset2_timing_plot.png}
    \caption{Dataset 2: Inference Speed}
\end{figure}

\begin{figure}[H]
    \centering
    \begin{subfigure}{0.45\textwidth}
        \centering
        \includegraphics[width=\textwidth]{figures/q3_c_no_cache_dset2_samples.png}
        \caption{Dataset 2: Samples (no caching)}
    \end{subfigure}
    \hspace{0.2in}
    \begin{subfigure}{0.45\textwidth}
        \centering
        \includegraphics[width=\textwidth]{figures/q3_c_with_cache_dset2_samples.png}
        \caption{Dataset 2: Samples (caching)}
    \end{subfigure}
\end{figure}
\end{enumerate}

%--------------------------------------------------------------------------------
%--------------------------------------------------------------------------------
%--------------------------------------------------------------------------------
\newpage
\noindent {\bf Question 4: Causal Transformer -- Tokenized Images}
%--------------------------------------------------------------------------------
%--------------------------------------------------------------------------------
%--------------------------------------------------------------------------------

\begin{enumerate}[(a)]
\item {\bf [5pt] Image Quantization}
\begin{figure}[H]
    \centering
    \begin{subfigure}{0.3\textwidth}
        \centering
        \includegraphics[width=\textwidth]{figures/q4_a_dset1_samples.png}
        \caption{Dataset 1: Quantized Examples}
    \end{subfigure}
    \hspace{0.2in}
    \begin{subfigure}{0.3\textwidth}
        \centering
        \includegraphics[width=\textwidth]{figures/q4_a_dset2_samples.png}
        \caption{Dataset 2: Quantized Examples}
    \end{subfigure} \\
\end{figure}

\item {\bf [15pt] Autoregressive Transformer on Colored Shapes and MNIST with Vector Quantization} \\\\
6.3M parameters for dataset 1 \\
Final test loss for dataset 1: 2.933 nats/dim
\begin{figure}[H]
    \centering
    \begin{subfigure}{0.4\textwidth}
        \centering
        \includegraphics[width=\textwidth]{figures/q4_b_dset1_train_plot.png}
        \caption{Dataset 1: Training curve}
    \end{subfigure}
    \hspace{0.2in}
    \begin{subfigure}{0.4\textwidth}
        \centering
        \includegraphics[width=\textwidth]{figures/q4_b_dset1_samples.png}
        \caption{Dataset 1: Samples}
    \end{subfigure}
\end{figure}
6.3M parameters for dataset 2 \\
Final test loss for dataset 2: 3.076 nats/dim
\begin{figure}[H]
    \centering
    \begin{subfigure}{0.4\textwidth}
        \centering
        \includegraphics[width=\textwidth]{figures/q4_b_dset2_train_plot.png}
        \caption{Dataset 2: Training curve}
    \end{subfigure}
    \hspace{0.2in}
    \begin{subfigure}{0.4\textwidth}
        \centering
        \includegraphics[width=\textwidth]{figures/q4_b_dset2_samples.png}
        \caption{Dataset 2: Samples}
    \end{subfigure}
\end{figure}

\end{enumerate}

%--------------------------------------------------------------------------------
%--------------------------------------------------------------------------------
%--------------------------------------------------------------------------------
\newpage
\noindent {\bf Question 5: Causal Transformer -- Text}
%--------------------------------------------------------------------------------
%--------------------------------------------------------------------------------
%--------------------------------------------------------------------------------

\begin{enumerate}[(a)]
\item {\bf [20pt] Modeling Text} \\\\
2.8M parameters \\
Final test loss: 1.828 nats/dim
\begin{figure}[H]
    \centering
\end{figure}

\begin{figure}[H]
    \centering
\end{figure}

\begin{figure}[H]
    \centering
    \begin{subfigure}{0.3\textwidth}
        \centering
        \includegraphics[width=\textwidth]{figures/q5_a_train_plot.png}
        \caption{Training curve}
    \end{subfigure}
    \hspace{0.2in}
    \begin{subfigure}{0.65\textwidth}
        \centering
        \includegraphics[width=\textwidth]{figures/q5_a_samples.png}
        \caption{Text samples}
    \end{subfigure}
\end{figure}

\end{enumerate}

%--------------------------------------------------------------------------------
%--------------------------------------------------------------------------------
%--------------------------------------------------------------------------------
\newpage
\noindent {\bf Question 6: Causal Transformer -- Multimodal}
%--------------------------------------------------------------------------------
%--------------------------------------------------------------------------------
%--------------------------------------------------------------------------------

\begin{enumerate}[(a)]
\item {\bf [20pt] Multimodal Text and Image Generation} \\\\
6.4M parameters \\
Final test loss: 2.617 nats/dim
\begin{figure}[H]
    \centering
    \begin{subfigure}{0.45\textwidth}
        \centering
        \includegraphics[width=\textwidth]{figures/q6_a_train_plot.png}
        \caption{Training curve}
    \end{subfigure}
    \hspace{0.2in}
    \begin{subfigure}{0.45\textwidth}
        \centering
        \includegraphics[width=\textwidth]{figures/q6_a_samples_img_conditioned.png}
        \caption{Image conditioned samples}
    \end{subfigure} \\
    \begin{subfigure}{0.45\textwidth}
        \centering
        \includegraphics[width=\textwidth]{figures/q6_a_samples_text_conditioned.png}
        \caption{Text conditioned samples}
    \end{subfigure}
    \hspace{0.2in}
    \begin{subfigure}{0.45\textwidth}
        \centering
        \includegraphics[width=\textwidth]{figures/q6_a_samples_unconditional.png}
        \caption{Unconditional samples}
    \end{subfigure}
\end{figure}

\end{enumerate}

\end{document}
